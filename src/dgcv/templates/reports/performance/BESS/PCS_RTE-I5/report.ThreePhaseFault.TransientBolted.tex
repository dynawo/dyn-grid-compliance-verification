    \subsection{Overview of DTR PCS \DTRPcs}

    Checks for compliant behavior of the PPM under a grid scenario where there
    is a symmetric three-phase fault at one of the four transmission lines
    (at a 1\% distance from the PDR). The PPM should remain connected and
    be able to supply the necessary reactive injection.

    The grid model and its operational point is as in the following schematic:
    \begin{center}
        \includestandalone[width=0.8\textwidth]{circuit_\DTRPcs.tikz}
    \end{center}
    \begin{center}
        \small \textbf{Note: This schematic is only a reminder of the test setup on the TSO's
        side --- the Producer's side may vary, depending on the user-provided model.}
    \end{center}

    \noindent Important reminders:
    \begin{itemize}
        \item The time for fault clearance, $T_{clear}$, is a parameter that
        is specified in the DTR and depends on the voltage level in
        question:
        \begin{itemize}
            \item HTB3 (400 kV): $T_{clear}$ = 85 ms
            \item HTB2 (150 kV and 225 kV): $T_{clear}$ = 85 ms
            \item HTB1 (63 kV and 90 kV): $T_{clear}$ = 150 ms
        \end{itemize}
    \end{itemize}

    In \Dynawo{} this fault has been modeled by splitting \code{Line4} into
    \code{Line4a} and \code{Line4b}, and then inserting an intermediate bus between
    them, to which a \code{NodeFault} object is attached.

    Note that although the DTR PCS also requires simulating a one-phase
    fault, this tool will not implement it, as it would require the
    simulation of a three-phase model. All simulations are only
    positive-sequence.


    \subsubsection{Simulation}
    As required by the DTR PCS \DTRPcs, the figures below show the
    following magnitudes:
    \begin{itemize}
        \item Voltage at connection point: modulus of the AC complex voltage at
        the PDR bus.
        \item Reactive power supplied at the connection point: sum of reactive power
        over all lines on TSO side.
        \item Active power supplied at the connection point: sum of active power
        over all lines on TSO side.
        \item The injected active and reactive currents. All PPM
        units are plotted on the same graph.
        \item Magnitude controlled by the AVR: modulus of the voltage at the REPC.
        All PPM units are plotted on the same graph.
        \item The PPM's main transformer tap ratio. if it is a transformer with an
        automatic tap changer.
    \end{itemize}

    \subsubsection{Simulation results}
    The blue line shows the calculated curve, if a reference curve has been entered it is
    shown in orange.

% For now we won't use floats for figures, to get more precise placement
    \noindent
    \begin{minipage}[t]{0.48\textwidth}
        \centering
        \includegraphics[width=\textwidth]{fig_V_PCS_RTE-I5.ThreePhaseFault.TransientBolted}
        \begin{minipage}[t]{0.8\textwidth}
            \small Voltage magnitude measured at the PDR bus.
            The vertical dashed line marks the rise time $T_{85U}$.
        \end{minipage}
    \end{minipage}
    \hfill
    \begin{minipage}[t]{0.48\textwidth}
        \centering
        \includegraphics[width=\textwidth]{fig_Ustator_PCS_RTE-I5.ThreePhaseFault.TransientBolted}
        \begin{minipage}[t]{0.8\textwidth}
            \small Magnitude controlled by the AVR: modulus of the voltage at the REPC, in pu.
            The gray dotted line show the AVR setpoint. All PPM units are plotted on the same
            graph.
        \end{minipage}
    \end{minipage}

    \vspace{0.5cm}

    \noindent
    \begin{minipage}[t]{0.48\textwidth}
        \centering
        \includegraphics[width=\textwidth]{fig_P_PCS_RTE-I5.ThreePhaseFault.TransientBolted}
        \begin{minipage}[t]{0.8\textwidth}
            \small Real power output P, measured at the PDR bus. Green-dashed
            line: $-10\%$ ``floor'' below the final value of P.
        \end{minipage}
    \end{minipage}
    \hfill
    \begin{minipage}[t]{0.48\textwidth}
        \centering
        \includegraphics[width=\textwidth]{fig_Q_PCS_RTE-I5.ThreePhaseFault.TransientBolted}
        \begin{minipage}[t]{0.8\textwidth}
            \small Reactive power output Q, measured at the PDR bus.
        \end{minipage}
    \end{minipage}

    \vspace{0.5cm}

    \noindent
    \begin{minipage}[t]{0.48\textwidth}
        \centering
        \includegraphics[width=\textwidth]{fig_I_PCS_RTE-I5.ThreePhaseFault.TransientBolted}
        \begin{minipage}[t]{0.8\textwidth}
            \small Injected active (light blue line) and reactive (violet line) currents. All PPM
            units are plotted on the same graph.
        \end{minipage}
    \end{minipage}
    \hfill
    \begin{minipage}[t]{0.48\textwidth}
        \centering
        \includegraphics[width=\textwidth]{fig_Tap_PCS_RTE-I5.ThreePhaseFault.TransientBolted}
        \begin{minipage}[t]{0.8\textwidth}
            \small The PPM's main transformer tap ratio (if it is a transformer with an
            automatic tap changer).
        \end{minipage}
    \end{minipage}
    \\[2\baselineskip]
    Go to  {{ linkPCSI5ThreePhaseFaultTransientBolted }}


    \subsubsection{Analysis of results}

    \noindent Analysis of results for PCS \DTRPcs. Values extracted
    from the simulation:

    \begin{center}
        \begin{tabular}{lc}
            \toprule
            \textbf{Parameter} & \multicolumn{1}{c}{\textbf{value}} \\
            \midrule
            \BLOCK{for row in rmPCSI5ThreePhaseFaultTransientBolted}
            {{row[0]}}         & {{row[1]}}                         \\
            \BLOCK{endfor}
            \bottomrule
        \end{tabular}
    \end{center}

    \noindent Key:
    \begin{description}
        \item[Recovery time $T_{10P_{floor}}$:] time at which the suplied power
        P recovers and stays above -10\% of the final value of P (note
        how this is not a "tube", but a "floor").
        \item[Rise time $T_{85U}$:] time at which the the voltage at the PDR
        bus returns back above 0.85 pu, regardless of any possible
        overshooting / undershooting that may take place later on.
    \end{description}


    \subsubsection{Compliance checks}

    Compliance checks required by PCS \DTRPcs:

    \begin{minipage}{\linewidth} % because otherwise, the footnote does not show
        \begin{tabular}{lcc}
            \toprule
            \textbf{Check} & \multicolumn{1}{c}{\textbf{value}} & \multicolumn{1}{c}{\textbf{Compliant?}} \\
            \midrule
            \BLOCK{for row in cmPCSI5ThreePhaseFaultTransientBolted}
            {{row[0]}}     & {{row[1]}}                         & {{row[2]}}                              \\
            \BLOCK{endfor}
            \bottomrule
        \end{tabular}
    \end{minipage}
