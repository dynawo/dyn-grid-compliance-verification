    \subsection{Overview of DTR PCS \DTRPcs}

    Checks for compliant behavior of the generator under a grid scenario where there is a
    sudden change in the admittance of the connection to the network (loss of one of
    the three transmission lines, i.e. a 33\% loss in branch admittance).

    The grid model and its operational point is as in the following schematic:
    \begin{center}
        \includestandalone[width=0.8\textwidth]{circuit_\DTRPcs.tikz}
    \end{center}
    \begin{center}
        \small \textbf{Note: This schematic is only a reminder of the test setup on the TSO's
        side --- the Producer's side may vary, depending on the user-provided model.}
    \end{center}


    \subsubsection{Simulation parameters}

    Solver and parameters used in the simulation:
    \begin{center}
        \begin{tabular}{lc}
            \toprule
           \textbf{Parameter} & \textbf{Value (default)} \\
            \midrule
            \BLOCK{for row in solverPCSI3LineTrip2BReactance}
            {{row[0]}}         & {{row[1]}}                         \\
            \BLOCK{endfor}
            \bottomrule
        \end{tabular}
    \end{center}

    \subsubsection{Simulation}
    As required by the DTR PCS \DTRPcs, the figures below show the
    following magnitudes:
    \begin{itemize}
        \item Voltage at connection point: modulus of the AC complex voltage at
        the PDR bus.
        \item Reactive power supplied at the connection point: sum of reactive power
        over all lines on TSO side.
        \item Active power supplied at the connection point: sum of active power
        over all lines on TSO side.
        \item Rotor speed.
        \item Magnitude controlled by the AVR.
    \end{itemize}

    \subsubsection{Simulation results}
    The blue line shows the calculated curve, if a reference curve has been entered it is
    shown in orange.

% For now we won't use floats for figures, to get more precise placement
    \noindent
    \begin{minipage}[t]{0.48\textwidth}
        \centering
        \includegraphics[width=\textwidth]{fig_V_PCS_RTE-I3.LineTrip.2BReactance}
        \begin{minipage}[t]{0.8\textwidth}
            \small Voltage magnitude measured at the PDR bus.
        \end{minipage}
    \end{minipage}
    \hfill
        \begin{minipage}[t]{0.48\textwidth}
            \centering
            \includegraphics[width=\textwidth]{fig_Ustator_PCS_RTE-I3.LineTrip.2BReactance}
            \begin{minipage}[t]{0.8\textwidth}
                \small Stator voltage magnitude, in pu. The gray dotted line show
                the AVR setpoint.
            \end{minipage}
        \end{minipage}

    \vspace{0.5cm}

    \noindent
    \begin{minipage}[t]{0.48\textwidth}
        \centering
        \includegraphics[width=\textwidth]{fig_P_PCS_RTE-I3.LineTrip.2BReactance}
        \begin{minipage}[t]{0.8\textwidth}
            \small Real power output P, measured at the PDR bus. Green-dashed
            lines: $\pm 10\%$ margin around the final value of P. Red-dashed
            lines: same, for $\pm 5\%$ margin.
        \end{minipage}
    \end{minipage}
    \hfill
    \begin{minipage}[t]{0.48\textwidth}
        \centering
        \includegraphics[width=\textwidth]{fig_Q_PCS_RTE-I3.LineTrip.2BReactance}
        \begin{minipage}[t]{0.8\textwidth}
            \small Reactive power output Q, measured at the PDR bus.
        \end{minipage}
    \end{minipage}

    \vspace{0.5cm}

    \begin{minipage}[t]{0.48\textwidth}
        \centering
        \includegraphics[width=\textwidth]{fig_W_PCS_RTE-I3.LineTrip.2BReactance}
        \begin{minipage}[t]{0.8\textwidth}
            \small Rotor speed.
        \end{minipage}
    \end{minipage}
    \\[2\baselineskip]
    Go to  {{ linkPCSI3LineTrip2BReactance }}


    \subsubsection{Analysis of results}

    \noindent Analysis of results for PCS \DTRPcs. Values extracted
    from the simulation:

    \begin{center}
        \begin{tabular}{lc}
            \toprule
            \textbf{Parameter} & \multicolumn{1}{c}{\textbf{value}} \\
            \midrule
            \BLOCK{for row in rmPCSI3LineTrip2BReactance}
            {{row[0]}}         & {{row[1]}}                         \\
            \BLOCK{endfor}
            \bottomrule
        \end{tabular}
    \end{center}

    \noindent Key:
    \begin{description}
        \item[Settling time $T_{10P}$:] time at which the suplied real power
        P stays within the +/-10\% tube centered on the final value of P.
        \item[Settling time $T_{5P}$:] time at which the suplied real power
        P stays within the +/-5\% tube centered on the final value of P.
    \end{description}


    \subsubsection{Compliance checks}

    Compliance checks required by PCS \DTRPcs:
    \begin{center}
        \begin{tabular}{lcc}
            \toprule
            \textbf{Check} & \multicolumn{1}{c}{\textbf{value}} & \multicolumn{1}{c}{\textbf{Compliant?}} \\
            \midrule
            \BLOCK{for row in cmPCSI3LineTrip2BReactance}
            {{row[0]}}     & {{row[1]}}                         & {{row[2]}}                              \\
            \BLOCK{endfor}
            \bottomrule
        \end{tabular}
    \end{center}
