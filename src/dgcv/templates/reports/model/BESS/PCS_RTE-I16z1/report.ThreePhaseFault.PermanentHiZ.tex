    %%%%%%%%%%%%%%%%%%%%%%%%%%%%%%%%%%%%%%%%%%%%%%%%%%%%%%%%%%%%%%%%%%%%%%%%%%%%%%%%%%%%%%%%%%%%%%%%%%%%%%%%%%%%%
    %%%%%%%%%%%%%%%%%%%%%%%%%%%%%%%%%%%%%%% Three Phase Fault %%%%%%%%%%%%%%%%%%%%%%%%%%%%%%%%%%%%%%%%%%%%%%%%%
    %%%%%%%%%%%%%%%%%%%%%%%%%%%%%%%%%%%%%%%%%%%%%%%%%%%%%%%%%%%%%%%%%%%%%%%%%%%%%%%%%%%%%%%%%%%%%%%%%%%%%%%%%%%%%

    \renewcommand{\DTRPcs}{ThreePhaseFault} % DTR pcs definition
    \renewcommand{\DTRPcsLong}{Three Phase Fault (Model Validation Zone 1)}
    \renewcommand{\OCname}{PermanentHiZ}


    \subsection{Permanent fault: High-impedance fault ($SCR=10, V=0.5Un$)}

    Checks for compliant behavior of the generating unit under a grid scenario where there
    is an impedance-to-earth three-phase permanent fault with a voltage dip of 0.5 $U_{dim}$.

    \GridCircuitZone

    \begin{description}
        \item Initial conditions used at the PDR bus:
        \begin{itemize}
            \item $P = P_\text{max\_unite}$
            \item $Q = 0$
            \item $U = U_\text{dim}$
            \item $SCR = 10$
        \end{itemize}
    \end{description}

    \GridCurvesZone
    \\[2\baselineskip]
    Go to  {{ linkPCSI16z1ThreePhaseFaultPermanentHiZ }}

    \subsubsection{Analysis of results}
    \begin{center}
        \scriptsize
        \begin{tabular}{lcccccc}
            \toprule
            & \multicolumn{3}{c}{Pre-event} & \multicolumn{3}{c}{Event} \\
            \cmidrule(lr){2-4}\cmidrule(lr){5-7}
            & {MXE}      & {ME}       & {MAE}      & {MXE}      & {ME}       & {MAE}      \\
            \midrule
            \BLOCK{for row in rmPCSI16z1ThreePhaseFaultPermanentHiZ}
            {{row[0]}} & {{row[1]}} & {{row[2]}} & {{row[3]}} & {{row[4]}} & {{row[5]}} & {{row[6]}} \\
            \BLOCK{endfor}
            \bottomrule
        \end{tabular}
    \end{center}

    \subsubsection{Compliance checks}

    \noindent Compliance thresholds on the curves:
    \begin{center}
        \scriptsize
        \begin{tabular}{lcccccc}
            \toprule
            & \multicolumn{3}{c}{Pre-event} & \multicolumn{3}{c}{Event} \\
            \cmidrule(lr){2-4}\cmidrule(lr){5-7}
            & {MXE}      & {ME}       & {MAE}      & {MXE}      & {ME}       & {MAE}      \\
            \midrule
            \BLOCK{for row in thmPCSI16z1ThreePhaseFaultPermanentHiZ}
            {{row[0]}} & {{row[1]}} & {{row[2]}} & {{row[3]}} & {{row[4]}} & {{row[5]}} & {{row[6]}} \\
            \BLOCK{endfor}
            \bottomrule
        \end{tabular}
    \end{center}

    \noindent Compliance checks on the curves:
    \begin{center}
        \scriptsize
        \begin{tabular}{lccccccc}
            \toprule
            & \multicolumn{3}{c}{Pre-event} & \multicolumn{3}{c}{Event} & \\
            \cmidrule(lr){2-4}\cmidrule(lr){5-7}
            & {MXE}      & {ME}       & {MAE}      & {MXE}      & {ME}       & {MAE}      & Compl.     \\
            \midrule
            \BLOCK{for row in emPCSI16z1ThreePhaseFaultPermanentHiZ}
            {{row[0]}} & {{row[1]}} & {{row[2]}} & {{row[3]}} & {{row[4]}} & {{row[5]}} & {{row[6]}} & {{row[7]}} \\
            \BLOCK{endfor}
            \bottomrule
        \end{tabular}
    \end{center}

    \noindent Compliance checks on the Active Power Recovery:
    \begin{center}
        \scriptsize
        \begin{tabular}{cllc}
            \toprule
            Variable   & Error      & Threshold   & Check      \\
            \midrule
            \BLOCK{for row in aprPCSI16z1ThreePhaseFaultPermanentHiZ}
            {{row[0]}} & {{row[1]}} & {{row[2]}}  & {{row[3]}} \\
            \BLOCK{endfor}
            \bottomrule
        \end{tabular}
    \end{center}

    \noindent In steady state after the event, the absolute average error must not exceed {{steadystatethreshold}}\% (configured value):
    \begin{center}
        \scriptsize
        \begin{tabular}{cllc}
            \toprule
            Variable   & MAE        & Final Error & Compliance \\
            \midrule
            \BLOCK{for row in ssemPCSI16z1ThreePhaseFaultPermanentHiZ}
            {{row[0]}} & {{row[1]}} & {{row[2]}}  & {{row[3]}} \\
            \BLOCK{endfor}
            \bottomrule
        \end{tabular}
    \end{center}
