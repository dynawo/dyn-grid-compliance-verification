    %%%%%%%%%%%%%%%%%%%%%%%%%%%%%%%%%%%%%%%%%%%%%%%%%%%%%%%%%%%%%%%%%%%%%%%%%%%%%%%%%%%%%%%%%%%%%%%%%%%%%%%%%%%%%%
    %%%%%%%%%%%%%%%%%%%%%%%%%%%%%%%%%%%%%%%%%%%%%%%%% PCS I6 %%%%%%%%%%%%%%%%%%%%%%%%%%%%%%%%%%%%%%%%%%%%%%%%%%%
    %%%%%%%%%%%%%%%%%%%%%%%%%%%%%%%%%%%%%%%%%%%%%%%%%%%%%%%%%%%%%%%%%%%%%%%%%%%%%%%%%%%%%%%%%%%%%%%%%%%%%%%%%%%%%%
    \renewcommand{\DTRPcs}{GridVoltageDip} % DTR pcs definition
    \renewcommand{\OCname}{Qzero}


    \subsection{Test I6 - Voltage sag ride-through}

    Checks for compliant behavior of the generator under a grid scenario where there is a
    severe drop of voltage at the PDR bus (which simulates the effect of a fault
    nearby in the network).

    \GridCircuitZthree

    \noindent Important reminders:
    \begin{itemize}
        \item Contrary to the modeling that the DTR suggests for this PCS, which uses a
        variable shunt impedance Zv and a short-circuit network impedance Zcc, here
        the voltage drop is simulated via a variable Infinite Bus source (Modelica
        model \code{ControllableSource}), directly attached to the PDR bus, and for
        which we specify a voltage curve that faithfully follows the shape prescribed
        by PCS I6.
    \end{itemize}

    In this PCS there are several events, which correspond to the changes that are
    forced upon the PDR bus voltage curve:
    \begin{itemize}
        \item At $t = T_{0}$, voltage drops to $U_{fault}$
        \item At $t = T_{clear}$, voltage rises to $U_{clear}$
        \item At $t = T_{rec1}$, voltage starts ramping up linearly, from $U_{clear}$ to
        $U_{rec2}$ (reached at $t = T_{rec2}$).
        \item At $t = T_{rec2}$ the voltage stops ramping up and remains at $U_{rec2}$
        indefinitely.
    \end{itemize}
    These values of time and voltage are specified by the DTR as follows. (Remember
    that the classification of generators into class A, B, C, D, is in the DTR under
    Chapter 5, Article 5.1.1, Section 3. See the reminder in the document
    \code{Summary\_data\_from\_the\_DTR.md}.)

    For PPMs of classes B, C, and Offshore (under 110 kV in this last case):
    \begin{center}
        \begin{tabular}{lc}
            \toprule
            \textbf{variable} & \textbf{all Volt Levels} \\
            \midrule
            $U_{initial}$     & 1.0 pu                   \\
            $T_{0}$           & (free)                   \\
            $U_{fault}$       & 0.05 pu                  \\
            $T_{clear}$       & $T_{0}$ + 150 ms         \\
            $T_{rec2}$        & $T_{0}$ + 1150 ms        \\
            $U_{rec2}$        & 0.85 pu                  \\
            \bottomrule
        \end{tabular}
    \end{center}

    For PPMs of class D and Offshore systems at 110 kV and above:
    \begin{center}
        \begin{tabular}{lc}
            \toprule
            \textbf{variable} & \textbf{all Volt levels} \\
            \midrule
            $U_{initial}$     & 1.0 pu                   \\
            $T_{0}$           & (free)                   \\
            $U_{fault}$       & 0.00 pu                  \\
            $T_{clear}$       & $T_{0}$ + 150 ms         \\
            $T_{rec2 }$       & $T_{0}$ + 1500 ms        \\
            $U_{rec2}$        & 0.85 pu                  \\
            \bottomrule
        \end{tabular}
    \end{center}

    \begin{description}
        \item Initial conditions used at the PDR bus:
        \begin{itemize}
            \item $P = P_{max\_unite}$
            \item $Q = 0$
            \item $U = U_{dim}$
        \end{itemize}
    \end{description}

    \subsubsection{Simulation parameters}

    Solver and parameters used in the simulation:
    \begin{center}
        \begin{tabular}{lc}
            \toprule
           \textbf{Parameter} & \textbf{Value (default)} \\
            \midrule
            \BLOCK{for row in solverPCSI16z3GridVoltageDipQzero}
            {{row[0]}}         & {{row[1]}}                         \\
            \BLOCK{endfor}
            \bottomrule
        \end{tabular}
    \end{center}

    \GridCurvesZthree
    \\[2\baselineskip]
    Go to  {{ linkPCSI16z3GridVoltageDipQzero }}

    \subsubsection{Analysis of results}

    \noindent Analysis of results for PCS \DTRPcs. Values extracted
    from the simulation:

    \begin{center}
        \scriptsize
        \begin{tabular}{@{}lccccccccc@{}}
            \toprule
            & \multicolumn{3}{c}{Pre-event} & \multicolumn{3}{c}{Event} & \multicolumn{3}{c}{Pos-Event} \\
            \cmidrule(lr){2-4}\cmidrule(lr){5-7}\cmidrule(lr){8-10}
            & {MXE}      & {ME}       & {MAE}      & {MXE}      & {ME}       & {MAE}      & {MXE}      & {ME}       & {MAE}      \\
            \midrule
            \BLOCK{for row in rmPCSI16z3GridVoltageDipQzero}
            {{row[0]}} & {{row[1]}} & {{row[2]}} & {{row[3]}} & {{row[4]}} & {{row[5]}} & {{row[6]}} & {{row[7]}} & {{row[8]}} & {{row[9]}} \\
            \BLOCK{endfor}
            \bottomrule
        \end{tabular}
    \end{center}

    \subsubsection{Compliance checks}

    \noindent Compliance thresholds on the curves:
    \begin{center}
        \scriptsize
        \begin{tabular}{@{}lccccccccc@{}}
            \toprule
            & \multicolumn{3}{c}{Pre-event} & \multicolumn{3}{c}{Event} & \multicolumn{3}{c}{Pos-Event} \\
            \cmidrule(lr){2-4}\cmidrule(lr){5-7}\cmidrule(lr){8-10}
            & {MXE}      & {ME}       & {MAE}      & {MXE}      & {ME}       & {MAE}      & {MXE}      & {ME}       & {MAE}      \\
            \midrule
            \BLOCK{for row in thmPCSI16z3GridVoltageDipQzero}
            {{row[0]}} & {{row[1]}} & {{row[2]}} & {{row[3]}} & {{row[4]}} & {{row[5]}} & {{row[6]}} & {{row[7]}} & {{row[8]}} & {{row[9]}} \\
            \BLOCK{endfor}
            \bottomrule
        \end{tabular}
    \end{center}

    \noindent Compliance checks on the curves:
    \begin{center}
        \scriptsize
        \begin{tabular}{@{}lcccccccccc@{}}
            \toprule
            & \multicolumn{3}{c}{Pre-event} & \multicolumn{3}{c}{Event} & \multicolumn{3}{c}{Pos-Event} & \\
            \cmidrule(lr){2-4}\cmidrule(lr){5-7}\cmidrule(lr){8-10}
            & {MXE}      & {ME}       & {MAE}      & {MXE}      & {ME}       & {MAE}      & {MXE}      & {ME}       & {MAE}      & Compl.      \\
            \midrule
            \BLOCK{for row in emPCSI16z3GridVoltageDipQzero}
            {{row[0]}} & {{row[1]}} & {{row[2]}} & {{row[3]}} & {{row[4]}} & {{row[5]}} & {{row[6]}} & {{row[7]}} & {{row[8]}} & {{row[9]}} & {{row[10]}} \\
            \BLOCK{endfor}
            \bottomrule
        \end{tabular}
    \end{center}

    \noindent Compliance checks on the Active Power Recovery:
    \begin{center}
        \scriptsize
        \begin{tabular}{cllc}
            \toprule
            Variable   & Error      & Threshold   & Check      \\
            \midrule
            \BLOCK{for row in aprPCSI16z3GridVoltageDipQzero}
            {{row[0]}} & {{row[1]}} & {{row[2]}}  & {{row[3]}} \\
            \BLOCK{endfor}
            \bottomrule
        \end{tabular}
    \end{center}

    \noindent In steady state after the event, the absolute average error must not exceed {{steadystatethreshold}}\% (configured value):
    \begin{center}
        \scriptsize
        \begin{tabular}{cllc}
            \toprule
            Variable   & MAE        & Final Error & Compliance \\
            \midrule
            \BLOCK{for row in ssemPCSI16z3GridVoltageDipQzero}
            {{row[0]}} & {{row[1]}} & {{row[2]}}  & {{row[3]}} \\
            \BLOCK{endfor}
            \bottomrule
        \end{tabular}
    \end{center}
