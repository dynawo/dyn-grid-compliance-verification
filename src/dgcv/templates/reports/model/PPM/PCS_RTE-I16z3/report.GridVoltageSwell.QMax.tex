    %%%%%%%%%%%%%%%%%%%%%%%%%%%%%%%%%%%%%%%%%%%%%%%%%%%%%%%%%%%%%%%%%%%%%%%%%%%%%%%%%%%%%%%%%%%%%%%%%%%%%%%%%%%%%%
    %%%%%%%%%%%%%%%%%%%%%%%%%%%%%%%%%%%%%%%%%%%%%%%%% PCS I7 QMax %%%%%%%%%%%%%%%%%%%%%%%%%%%%%%%%%%%%%%%%%%%%%%
    %%%%%%%%%%%%%%%%%%%%%%%%%%%%%%%%%%%%%%%%%%%%%%%%%%%%%%%%%%%%%%%%%%%%%%%%%%%%%%%%%%%%%%%%%%%%%%%%%%%%%%%%%%%%%%
    \renewcommand{\DTRPcs}{GridVoltageSwell} % DTR pcs definition
    \renewcommand{\OCname}{QMax}


    \subsection{Test I7 - Voltage swell ride-through ($Q_{max}$)}

    Checks for compliant behavior of the generator under a grid scenario where
    there is a severe overvoltage at the PDR bus (which simulates the
    effect of clearing a fault nearby in the network).

    \GridCircuitZthree

    In this PCS there are several events, which correspond to the changes that are
    forced upon the PDR bus voltage curve:
    \begin{itemize}
        \item At $t = T_{0}$, the voltage swells to $U_{swell}$
        \item At $t = T_{rec1}$, the voltage drops to $U_{rec2}$
        \item At $t = T_{rec2}$, the voltage drops to $U_{rec3}$
        \item At $t = T_{rec3}$, the voltage drops to $U_{rec4}$
    \end{itemize}
    These values of time and voltage are specified by the DTR as follows. (Remember
    that the classification of generators into class A, B, C, D, is in the DTR under
    Chapter 5, Article 5.1.1, Section 3. See the reminder in the document
    \code{Summary\_data\_from\_the\_DTR.md}.)

    \begin{center}
        \begin{tabular}{lc}
            \toprule
            \textbf{variable} & \textbf{all Volt Levels} \\
            \midrule
            $U_{initial}$     & 1.0 pu                   \\
            $T_{0}$           & (free)                   \\
            $U_{swell}$       & 1.3 pu                   \\
            $T_{rec1}$        & $T_{0}$ + 100 ms          \\
            $U_{rec2}$        & 1.25 pu                  \\
            $T_{rec2}$        & $T_{0}$ + 2.5 s          \\
            $U_{rec3}$        & 1.15 pu                  \\
            $T_{rec3}$        & $T_{0}$ + 30 s           \\
            $U_{rec4}$        & 1.10 pu                  \\
            \bottomrule
        \end{tabular}
    \end{center}

    \begin{description}
        \item Initial conditions used at the PDR bus:
        \begin{itemize}
            \item $P = P_{max\_unite}$
            \item $Q = Q_{max}$
            \item $U = U_{dim}$
        \end{itemize}
    \end{description}


    \subsubsection{Simulation parameters}

    Solver and parameters used in the simulation:
    \begin{center}
        \begin{tabular}{lc}
            \toprule
           \textbf{Parameter} & \textbf{Value (default)} \\
            \midrule
            \BLOCK{for row in solverPCSI16z3GridVoltageSwellQMax}
            {{row[0]}}         & {{row[1]}}                         \\
            \BLOCK{endfor}
            \bottomrule
        \end{tabular}
    \end{center}

    \GridCurvesZthree
    \\[2\baselineskip]
    Go to  {{ linkPCSI16z3GridVoltageSwellQMax }}

    \subsubsection{Analysis of results}

    \begin{center}
        \scriptsize
        \begin{tabular}{@{}lccccccccc@{}}
            \toprule
            & \multicolumn{3}{c}{Pre-event} & \multicolumn{3}{c}{Event} & \multicolumn{3}{c}{Pos-Event} \\
            \cmidrule(lr){2-4}\cmidrule(lr){5-7}\cmidrule(lr){8-10}
            & {MXE}      & {ME}       & {MAE}      & {MXE}      & {ME}       & {MAE}      & {MXE}      & {ME}       & {MAE}      \\
            \midrule
            \BLOCK{for row in rmPCSI16z3GridVoltageSwellQMax}
            {{row[0]}} & {{row[1]}} & {{row[2]}} & {{row[3]}} & {{row[4]}} & {{row[5]}} & {{row[6]}} & {{row[7]}} & {{row[8]}} & {{row[9]}} \\
            \BLOCK{endfor}
            \bottomrule
        \end{tabular}
    \end{center}

    \subsubsection{Compliance checks}

    \noindent Compliance thresholds on the curves:
    \begin{center}
        \scriptsize
        \begin{tabular}{@{}lccccccccc@{}}
            \toprule
            & \multicolumn{3}{c}{Pre-event} & \multicolumn{3}{c}{Event} & \multicolumn{3}{c}{Pos-Event} \\
            \cmidrule(lr){2-4}\cmidrule(lr){5-7}\cmidrule(lr){8-10}
            & {MXE}      & {ME}       & {MAE}      & {MXE}      & {ME}       & {MAE}      & {MXE}      & {ME}       & {MAE}      \\
            \midrule
            \BLOCK{for row in thmPCSI16z3GridVoltageSwellQMax}
            {{row[0]}} & {{row[1]}} & {{row[2]}} & {{row[3]}} & {{row[4]}} & {{row[5]}} & {{row[6]}} & {{row[7]}} & {{row[8]}} & {{row[9]}} \\
            \BLOCK{endfor}
            \bottomrule
        \end{tabular}
    \end{center}

    \noindent Compliance checks on the curves:
    \begin{center}
        \scriptsize
        \begin{tabular}{@{}lcccccccccc@{}}
            \toprule
            & \multicolumn{3}{c}{Pre-event} & \multicolumn{3}{c}{Event} & \multicolumn{3}{c}{Pos-Event} & \\
            \cmidrule(lr){2-4}\cmidrule(lr){5-7}\cmidrule(lr){8-10}
            & {MXE}      & {ME}       & {MAE}      & {MXE}      & {ME}       & {MAE}      & {MXE}      & {ME}       & {MAE}      & Compl.      \\
            \midrule
            \BLOCK{for row in emPCSI16z3GridVoltageSwellQMax}
            {{row[0]}} & {{row[1]}} & {{row[2]}} & {{row[3]}} & {{row[4]}} & {{row[5]}} & {{row[6]}} & {{row[7]}} & {{row[8]}} & {{row[9]}} & {{row[10]}} \\
            \BLOCK{endfor}
            \bottomrule
        \end{tabular}
    \end{center}

    \noindent In steady state after the event, the absolute average error must not exceed {{steadystatethreshold}}\% (configured value):
    \begin{center}
        \scriptsize
        \begin{tabular}{cllc}
            \toprule
            Variable   & MAE        & Final Error & Compliance \\
            \midrule
            \BLOCK{for row in ssemPCSI16z3GridVoltageSwellQMax}
            {{row[0]}} & {{row[1]}} & {{row[2]}}  & {{row[3]}} \\
            \BLOCK{endfor}
            \bottomrule
        \end{tabular}

    \end{center}
