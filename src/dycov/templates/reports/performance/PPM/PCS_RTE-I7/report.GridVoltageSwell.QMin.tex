    \subsection{Overview of DTR PCS \DTRPcs{} ($Q_\text{PDR} = Q_\text{min}$)}

    Checks for compliant behavior of the PPM under a grid scenario where
    there is a severe overvoltage at the PDR bus (which simulates the
    effect of clearing a fault nearby in the network).

    The grid model and its operational point is as in the following schematic:
    \begin{center}
        \includestandalone[width=0.8\textwidth]{circuit_\DTRPcs.tikz}
    \end{center}
    \begin{center}
        \small \textbf{Note: This schematic is only a reminder of the test setup on the TSO's
        side --- the Producer's side may vary, depending on the user-provided model.}
    \end{center}
    where $Q_\text{PDR} = Q_\text{min}$.

    \noindent Important reminders:
    \begin{itemize}
        \item Contrary to the modeling that the DTR suggests for this PCS, which uses a
        variable shunt impedance Zv, here the voltage drop is simulated via a variable
        Infinite Bus source (Modelica model \code{ControllableSource}), directly
        attached to the PDR bus, and for which we specify a voltage curve that
        faithfully follows the shape prescribed by PCS I7.
    \end{itemize}

    In this PCS there are several events, which correspond to the changes that are
    forced upon the PDR bus voltage curve:
    \begin{itemize}
        \item At $t = T_{0}$, the voltage swells to $U_{swell}$
        \item At $t = T_{rec1}$, the voltage drops to $U_{rec2}$
        \item At $t = T_{rec2}$, the voltage drops to $U_{rec3}$
        \item At $t = T_{rec3}$, the voltage drops to $U_{rec4}$
    \end{itemize}
    These values of time and voltage are specified by the DTR as follows. (Remember
    that the classification of generators into class A, B, C, D, is in the DTR under
    Chapter 5, Article 5.1.1, Section 3. See the reminder in the document
    \code{Summary\_data\_from\_the\_DTR.md}.)

    \begin{center}
        \begin{tabular}{lc}
            \toprule
            \textbf{variable} & \textbf{all Volt Levels} \\
            \midrule
            $U_{initial}$     & 1.0 pu                   \\
            $T_{0}$           & (free)                   \\
            $U_{swell}$       & 1.3 pu                   \\
            $T_{rec1}$        & $T_{0}$ + 50 ms          \\
            $U_{rec2}$        & 1.25 pu                  \\
            $T_{rec2}$        & $T_{0}$ + 2.5 s          \\
            $U_{rec3}$        & 1.15 pu                  \\
            $T_{rec3}$        & $T_{0}$ + 30 s           \\
            $U_{rec4}$        & 1.10 pu                  \\
            \bottomrule
        \end{tabular}
    \end{center}

    \begin{description}
        \item Initial conditions used at the PDR bus:
        \begin{itemize}
            \item $P = P_{max\_unite}$
            \item $Q = Q_{min}$
            \item $U = U_{dim}$
        \end{itemize}
    \end{description}

    \subsubsection{Simulation parameters}

    Solver and parameters used in the simulation:
    \begin{center}
        \begin{tabular}{lc}
            \toprule
           \textbf{Parameter} & \textbf{Value (default)} \\
            \midrule
            \BLOCK{for row in solverPCSI7GridVoltageSwellQMin}
            {{row[0]}}         & {{row[1]}}                         \\
            \BLOCK{endfor}
            \bottomrule
        \end{tabular}
    \end{center}

    \subsubsection{Simulation}
    The following figures show the magnitudes that DTR PCS \DTRPcs
    requires to be monitored:
    \begin{itemize}
        \item Voltage at connection point: modulus of the AC complex voltage at
        the PDR bus.
        \item Reactive power supplied at the connection point (obtained from
        the controllable Infinite Bus on TSO side).
        \item Active power supplied at the connection point (obtained from
        the controllable Infinite Bus on TSO side).
        \item The injected active and reactive currents. All PPM
        units are plotted on the same graph.
        \item Magnitude controlled by the AVR: modulus of the voltage at the REPC.
        All PPM units are plotted on the same graph.
    \end{itemize}

    \subsubsection{Simulation results}
    The blue line shows the calculated curve, if a reference curve has been entered it is
    shown in orange.

% For now we won't use floats for figures, to get more precise placement
    \noindent
    \begin{minipage}[t]{0.48\textwidth}
        \centering
        \includegraphics[width=\textwidth]{fig_V_PCS_RTE-I7.GridVoltageSwell.QMin}
        \begin{minipage}[t]{0.8\textwidth}
            \small Voltage magnitude measured at the PDR bus.
        \end{minipage}
    \end{minipage}
    \hfill
        \begin{minipage}[t]{0.48\textwidth}
            \centering
            \includegraphics[width=\textwidth]{fig_Ustator_PCS_RTE-I7.GridVoltageSwell.QMin}
            \begin{minipage}[t]{0.8\textwidth}
                \small Magnitude controlled by the AVR: modulus of the voltage at the REPC, in pu.
                The gray dotted line show the AVR setpoint. All PPM units are plotted on the same
                graph.
            \end{minipage}
        \end{minipage}

    \vspace{0.5cm}

    \noindent
    \begin{minipage}[t]{0.48\textwidth}
        \centering
        \includegraphics[width=\textwidth]{fig_P_PCS_RTE-I7.GridVoltageSwell.QMin}
        \begin{minipage}[t]{0.8\textwidth}
            \small Real power output P, measured at the PDR bus.
        \end{minipage}
    \end{minipage}
    \hfill
    \begin{minipage}[t]{0.48\textwidth}
        \centering
        \includegraphics[width=\textwidth]{fig_Q_PCS_RTE-I7.GridVoltageSwell.QMin}
        \begin{minipage}[t]{0.8\textwidth}
            \small Reactive power output Q, measured at the PDR bus.
        \end{minipage}
    \end{minipage}

    \vspace{0.5cm}

    \begin{minipage}[t]{0.48\textwidth}
        \centering
        \includegraphics[width=\textwidth]{fig_I_PCS_RTE-I7.GridVoltageSwell.QMin}
        \begin{minipage}[t]{0.8\textwidth}
            \small Injected active (light blue line) and reactive (violet line) currents. All PPM
            units are plotted on the same graph.
        \end{minipage}
    \end{minipage}
    \\[2\baselineskip]
    Go to  {{ linkPCSI7GridVoltageSwellQMin }}


    \subsubsection{Analysis of results}

    \noindent No analysis is needed in PCS \DTRPcs.


    \subsubsection{Compliance checks}

    Compliance checks required by PCS \DTRPcs:
    \begin{center}
        \begin{tabular}{lcc}
            \toprule
            \textbf{Check} & \multicolumn{1}{c}{\textbf{value}} & \multicolumn{1}{c}{\textbf{Compliant?}} \\
            \midrule
            \BLOCK{for row in cmPCSI7GridVoltageSwellQMin}
            {{row[0]}}     & {{row[1]}}                         & {{row[2]}}                              \\
            \BLOCK{endfor}
            \bottomrule
        \end{tabular}
    \end{center}
