    \subsection{Overview of DTR PCS \DTRPcs}

    Checks for compliant behavior of the generator under a grid scenario where there is a
    severe drop of voltage at the PDR bus (which simulates the effect of a fault
    nearby in the network).

    The grid model and its operational point is as in the following schematic:
    \begin{center}
        \includestandalone[width=0.8\textwidth]{circuit_\DTRPcs.tikz}
    \end{center}
    \begin{center}
        \small \textbf{Note: This schematic is only a reminder of the test setup on the TSO's
        side --- the Producer's side may vary, depending on the user-provided model.}
    \end{center}

    \noindent Important reminders:
    \begin{itemize}
        \item Contrary to the modeling that the DTR suggests for this PCS, which uses a
        variable shunt impedance Zv and a short-circuit network impedance Zcc, here
        the voltage drop is simulated via a variable Infinite Bus source (Modelica
        model \code{ControllableSource}), directly attached to the PDR bus, and for
        which we specify a voltage curve that faithfully follows the shape prescribed
        by PCS I6.
    \end{itemize}

    In this PCS there are several events, which correspond to the changes that are
    forced upon the PDR bus voltage curve:
    \begin{itemize}
        \item At $t = T_{0}$, voltage drops to $U_{fault}$
        \item At $t = T_{clear}$, voltage rises to $U_{clear}$
        \item At $t = T_{rec1}$, voltage starts ramping up linearly, from $U_{clear}$ to
        $U_{rec2}$ (reached at $t = T_{rec2}$).
        \item At $t = T_{rec2}$ the voltage stops ramping up and remains at $U_{rec2}$
        indefinitely.
    \end{itemize}
    These values of time and voltage are specified by the DTR as follows. (Remember
    that the classification of generators into class A, B, C, D, is in the DTR under
    Chapter 5, Article 5.1.1, Section 3. See the reminder in the document
    \code{Summary\_data\_from\_the\_DTR.md}.)

    For generator units of classes B and C:
    \begin{center}
        \begin{tabular}{lcc}
            \toprule
            \textbf{variable} & \textbf{for HTB1} & \textbf{other Volt Levels} \\
            \midrule
            $U_{initial}$     & 1.0 pu            & 1.0 pu                     \\
            $T_{0}$           & (free)            & (free)                     \\
            $U_{fault}$       & 0.05 pu           & 0.05 pu                    \\
            $T_{clear}$       & $T_{0}$ + 250 ms  & $T_{0}$ + 150 ms           \\
            $U_{clear}$       & 0.7 pu            & 0.7 pu                     \\
            $T_{rec1 }$       & $T_{0}$ + 750 ms  & $T_{0}$ + 700 ms           \\
            $T_{rec2 }$       & $T_{0}$ + 1550 ms & $T_{0}$ + 1500 ms          \\
            $U_{rec2}$        & 0.9 pu            & 0.9 pu                     \\
            \bottomrule
        \end{tabular}
    \end{center}

    For generator units of class D:
    \begin{center}
        \begin{tabular}{lcc}
            \toprule
            \textbf{variable} & \textbf{for HTB2} & \textbf{other Volt Levels} \\
            \midrule
            $U_{initial}$     & 1.0 pu            & 1.0 pu                     \\
            $T_{0}$           & (free)            & (free)                     \\
            $U_{fault}$       & 0.00 pu           & 0.00 pu                    \\
            $T_{clear}$       & $T_{0}$ + 250 ms  & $T_{0}$ + 150 ms           \\
            $U_{clear}$       & 0.5 pu            & 0.5 pu                     \\
            $T_{rec1 }$       & $T_{0}$ + 750 ms  & $T_{0}$ + 700 ms           \\
            $T_{rec2 }$       & $T_{0}$ + 1400 ms & $T_{0}$ + 1350 ms          \\
            $U_{rec2}$        & 0.9 pu            & 0.9 pu                     \\
            \bottomrule
        \end{tabular}
    \end{center}

    \begin{description}
        \item Initial conditions used at the PDR bus:
        \begin{itemize}
            \item $P = P_{max\_unite}$
            \item $Q = 0$
            \item $U = U_{dim}$
        \end{itemize}
    \end{description}

    \subsubsection{Simulation parameters}

    Solver and parameters used in the simulation:
    \begin{center}
        \begin{tabular}{lc}
            \toprule
           \textbf{Parameter} & \textbf{Value (default)} \\
            \midrule
            \BLOCK{for row in solverPCSI6GridVoltageDipQzero}
            {{row[0]}}         & {{row[1]}}                         \\
            \BLOCK{endfor}
            \bottomrule
        \end{tabular}
    \end{center}

    \subsubsection{Simulation}
    As required by the DTR PCS \DTRPcs, the figures below show the
    following magnitudes:
    \begin{itemize}
        \item Voltage at connection point: modulus of the AC complex voltage at
        the PDR bus.
        \item Reactive power supplied at the connection point (obtained from
        the controllable Infinite Bus on TSO side).
        \item Active power supplied at the connection point (obtained from
        the controllable Infinite Bus on TSO side).
        \item Rotor speed.
        \item Magnitude controlled by the AVR.
        \item Generator's internal angle.
    \end{itemize}

    \subsubsection{Simulation results}
    The blue line shows the calculated curve, if a reference curve has been entered it is
    shown in orange.

% For now we won't use floats for figures, to get more precise placement
    \noindent
    \begin{minipage}[t]{0.48\textwidth}
        \centering
        \includegraphics[width=\textwidth]{\Producer_fig_V_PCS_RTE-I6.GridVoltageDip.Qzero}
        \begin{minipage}[t]{0.8\textwidth}
            \small Voltage magnitude measured at the PDR bus.
            The vertical dashed line marks the rise time $T_{85U}$.
        \end{minipage}
    \end{minipage}
    \hfill
    \begin{minipage}[t]{0.48\textwidth}
        \centering
        \includegraphics[width=\textwidth]{\Producer_fig_Ustator_PCS_RTE-I6.GridVoltageDip.Qzero}
        \begin{minipage}[t]{0.8\textwidth}
            \small Stator voltage magnitude, in pu. The gray dotted line show
            the AVR setpoint.
        \end{minipage}
    \end{minipage}

    \vspace{0.5cm}

    \noindent
    \begin{minipage}[t]{0.48\textwidth}
        \centering
        \includegraphics[width=\textwidth]{\Producer_fig_P_PCS_RTE-I6.GridVoltageDip.Qzero}
        \begin{minipage}[t]{0.8\textwidth}
            \small Real power output P, measured at the PDR bus. Green-dashed
            lines: $\pm 10\%$ margin around the final value of P.
        \end{minipage}
    \end{minipage}
    \hfill
    \begin{minipage}[t]{0.48\textwidth}
        \centering
        \includegraphics[width=\textwidth]{\Producer_fig_Q_PCS_RTE-I6.GridVoltageDip.Qzero}
        \begin{minipage}[t]{0.8\textwidth}
            \small Reactive power output Q, measured at the PDR bus.
        \end{minipage}
    \end{minipage}

    \vspace{0.5cm}

    \noindent
    \begin{minipage}[t]{0.48\textwidth}
        \centering
        \includegraphics[width=\textwidth]{\Producer_fig_W_PCS_RTE-I6.GridVoltageDip.Qzero}
        \begin{minipage}[t]{0.8\textwidth}
            \small Rotor speed.
        \end{minipage}
    \end{minipage}
    \hfill
    \begin{minipage}[t]{0.48\textwidth}
        \centering
        \includegraphics[width=\textwidth]{\Producer_fig_Theta_PCS_RTE-I6.GridVoltageDip.Qzero}
        \begin{minipage}[t]{0.8\textwidth}
            \small Generator's internal angle, in rad.
        \end{minipage}
    \end{minipage}
    \\[2\baselineskip]
    Go to  {{ linkPCSI6GridVoltageDipQzero }}


    \subsubsection{Analysis of results}

    \noindent Analysis of results for PCS \DTRPcs. Values extracted
    from the simulation:

    \begin{center}
        \begin{tabular}{lc}
            \toprule
            \textbf{Parameter} & \multicolumn{1}{c}{\textbf{value}} \\
            \midrule
            \BLOCK{for row in rmPCSI6GridVoltageDipQzero}
            {{row[0]}}         & {{row[1]}}                         \\
            \BLOCK{endfor}
            \bottomrule
        \end{tabular}
    \end{center}

    \noindent Key:
    \begin{description}
        \item[Settling time $T_{10P}$:] time at which the suplied real power
        P stays within the +/-10\% tube centered on the final value of P.
        \item[Rise time $T_{85U}$:] time at which the the voltage at the
        PDR bus returns back above 0.85pu, regardless of any possible
        overshooting/undershooting that may take place later on.
    \end{description}


    \subsubsection{Compliance checks}

    Compliance checks required by PCS \DTRPcs:
    \begin{center}
        \begin{tabular}{lcc}
            \toprule
            \textbf{Check} & \multicolumn{1}{c}{\textbf{value}} & \multicolumn{1}{c}{\textbf{Compliant?}} \\
            \midrule
            \BLOCK{for row in cmPCSI6GridVoltageDipQzero}
            {{row[0]}}     & {{row[1]}}                         & {{row[2]}}                              \\
            \BLOCK{endfor}
            \bottomrule
        \end{tabular}
    \end{center}
