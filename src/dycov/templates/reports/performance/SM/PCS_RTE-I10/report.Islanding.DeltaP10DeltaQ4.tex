    \subsection{Overview of DTR PCS \DTRPcs}

    Checks for compliant behavior of the generator under an
    \emph{islanding} event grid scenario, i.e., the network has been split in
    such a way that the generator is left as the only synchronous machine
    that should sustain the frequency of its subnetwork. The generator
    should maintain the stability until the restoration process
    re-synchronizes this subnetwork to the bulk transmission network.

    The grid model and its operational point is as in the following schematic:
    \begin{center}
        \includestandalone[width=0.8\textwidth]{circuit_\DTRPcs.tikz}
    \end{center}
    \begin{center}
        \small \textbf{Note: This schematic is only a reminder of the test setup on the TSO's
        side --- the Producer's side may vary, depending on the user-provided model.}
    \end{center}

    In this case the network is not modeled. In Dynawo, we simply attach
    two loads of type \code{LoadPQ} (since the DTR specifies that the
    loads have alpha=beta=0 and do not depend on frequency). Then their
    PQ values are configured as follows:
    \begin{itemize}
        \item Load 1: P = 0.7 Pmax, Q = -0.04 Pmax
        \item Load 2: P = 0.1 Pmax, Q = 0.04 Pmax
    \end{itemize}
    And then disconnect Load 2. This complies with the initial conditions
    and the desired changes specified by the DTR.

    \begin{description}
        \item Initial conditions used at the PDR bus:
        \begin{itemize}
            \item $P = 0.8 P_{max\_unite}$
            \item $Q = 0$
            \item $U = U_{dim}$
            \item $f = 50 Hz$ (this is always the default in Dynawo)
        \end{itemize}
    \end{description}

    \subsubsection{Simulation parameters}

    Solver and parameters used in the simulation:
    \begin{center}
        \begin{tabular}{lc}
            \toprule
           \textbf{Parameter} & \textbf{Value (default)} \\
            \midrule
            \BLOCK{for row in solverPCSI10IslandingDeltaP10DeltaQ4}
            {{row[0]}}         & {{row[1]}}                         \\
            \BLOCK{endfor}
            \bottomrule
        \end{tabular}
    \end{center}

    \subsubsection{Simulation}
    As required by the DTR PCS \DTRPcs, the figures below show the
    following magnitudes:
    \begin{itemize}
        \item Voltage at connection point: modulus of the AC complex voltage at
        the PDR bus.
        \item Reactive power supplied at the connection point: sum of reactive power over
        all loads connected at the PDR bus on TSO side.
        \item Active power supplied at the connection point: sum of active power over
        all loads connected at the PDR bus on TSO side.
        \item Rotor speed.
        \item Magnitude controlled by the AVR.
        \item Network frequency.
    \end{itemize}

    \subsubsection{Simulation results}
    The blue line shows the calculated curve, if a reference curve has been entered it is
    shown in orange.

% For now we won't use floats for figures, to get more precise placement
    \noindent
    \begin{minipage}[t]{0.48\textwidth}
        \centering
        \includegraphics[width=\textwidth]{fig_V_PCS_RTE-I10.Islanding.DeltaP10DeltaQ4}
        \begin{minipage}[t]{0.8\textwidth}
            \small Voltage magnitude measured at the PDR bus.
        \end{minipage}
    \end{minipage}
    \hfill
    \begin{minipage}[t]{0.48\textwidth}
        \centering
        \includegraphics[width=\textwidth]{fig_Ustator_PCS_RTE-I10.Islanding.DeltaP10DeltaQ4}
        \begin{minipage}[t]{0.8\textwidth}
            \small Stator voltage magnitude, in pu. The gray dotted line show
            the AVR setpoint.
        \end{minipage}
    \end{minipage}

    \vspace{0.5cm}

    \noindent
    \begin{minipage}[t]{0.48\textwidth}
        \centering
        \includegraphics[width=\textwidth]{fig_P_PCS_RTE-I10.Islanding.DeltaP10DeltaQ4}
        \begin{minipage}[t]{0.8\textwidth}
            \small Real power output P, measured at the PDR bus.
        \end{minipage}
    \end{minipage}
    \hfill
    \begin{minipage}[t]{0.48\textwidth}
        \centering
        \includegraphics[width=\textwidth]{fig_Q_PCS_RTE-I10.Islanding.DeltaP10DeltaQ4}
        \begin{minipage}[t]{0.8\textwidth}
            \small Reactive power output Q, measured at the PDR bus.
        \end{minipage}
    \end{minipage}

    \vspace{0.5cm}

    \noindent
    \begin{minipage}[t]{0.48\textwidth}
        \centering
        \includegraphics[width=\textwidth]{fig_W_PCS_RTE-I10.Islanding.DeltaP10DeltaQ4}
        \begin{minipage}[t]{0.8\textwidth}
            \small Rotor speed.
        \end{minipage}
    \end{minipage}
    \hfill
    \begin{minipage}[t]{0.48\textwidth}
        \centering
        \includegraphics[width=\textwidth]{fig_WRef_PCS_RTE-I10.Islanding.DeltaP10DeltaQ4}
        \begin{minipage}[t]{0.8\textwidth}
            \small Network frequency, in pu. The red dashed horizontal lines mark
            the compliance limits, [0.98, 1.02] pu.
        \end{minipage}
    \end{minipage}
    \\[2\baselineskip]
    Go to  {{ linkPCSI10IslandingDeltaP10DeltaQ4 }}


    \subsubsection{Analysis of results}

    \noindent No analysis is needed in PCS \DTRPcs.


    \subsubsection{Compliance checks}

    Compliance checks required by PCS \DTRPcs:

    \begin{minipage}{\linewidth} % because otherwise, the footnote does not show
        \begin{tabular}{lcc}
            \toprule
            \textbf{Check} & \multicolumn{1}{c}{\textbf{value}} & \multicolumn{1}{c}{\textbf{Compliant?}} \\
            \midrule
            \BLOCK{for row in cmPCSI10IslandingDeltaP10DeltaQ4}
            {{row[0]}}     & {{row[1]}}                         & {{row[2]}}                              \\
            \BLOCK{endfor}
            \bottomrule
        \end{tabular}
    \end{minipage}
