    %%%%%%%%%%%%%%%%%%%%%%%%%%%%%%%%%%%%%%%%%%%%%%%%%%%%%%%%%%%%%%%%%%%%%%%%%%%%%%%%%%%%%%%%%%%%%%%%%%%%%%%%%%%%%%
    %%%%%%%%%%%%%%%%%%%%%%%%%%%%%%%%%%%%%%%%%%%%%%%%% PCS I10 %%%%%%%%%%%%%%%%%%%%%%%%%%%%%%%%%%%%%%%%%%%%%%%%%%
    %%%%%%%%%%%%%%%%%%%%%%%%%%%%%%%%%%%%%%%%%%%%%%%%%%%%%%%%%%%%%%%%%%%%%%%%%%%%%%%%%%%%%%%%%%%%%%%%%%%%%%%%%%%%%%
    \renewcommand{\DTRPcs}{Islanding} % DTR pcs definition
    \renewcommand{\OCname}{DeltaP10DeltaQ4}


    \subsection{Test I10 - Islanding event}

    Checks for compliant behavior of the generator under an
    \emph{islanding} event grid scenario, i.e., the network has been split in
    such a way that the generator is left as the only synchronous machine
    that should sustain the frequency of its subnetwork. The generator
    should maintain the stability until the restoration process
    re-synchronizes this subnetwork to the bulk transmission network.

    \GridCircuitZthree

    In this case the network is not modeled. In Dynawo, we simply attach
    two loads of type \code{LoadPQ} (since the DTR specifies that the
    loads have alpha=beta=0 and do not depend on frequency). Then their
    PQ values are configured as follows:
    \begin{itemize}
        \item Load 1: P = 0.7 Pmax, Q = -0.04 Pmax
        \item Load 2: P = 0.1 Pmax, Q = 0.04 Pmax
    \end{itemize}
    And then disconnect Load 2. This complies with the initial conditions
    and the desired changes specified by the DTR.

    \begin{description}
        \item Initial conditions used at the PDR bus:
        \begin{itemize}
            \item $P = 0.8 P_{max\_unite}$
            \item $Q = 0$
            \item $U = U_{dim}$
            \item $f = 50 Hz$ (this is always the default in Dynawo)
        \end{itemize}
    \end{description}

    \subsubsection{Simulation parameters}

    Solver and parameters used in the simulation:
    \begin{center}
        \begin{tabular}{lc}
            \toprule
           \textbf{Parameter} & \textbf{Value (default)} \\
            \midrule
            \BLOCK{for row in solverPCSI16z3IslandingDeltaP10DeltaQ4}
            {{row[0]}}         & {{row[1]}}                         \\
            \BLOCK{endfor}
            \bottomrule
        \end{tabular}
    \end{center}

    \GridCurvesZthree
    \\[2\baselineskip]
    Go to  {{ linkPCSI16z3IslandingDeltaP10DeltaQ4 }}

    \subsubsection{Analysis of results}

%\noindent No analysis is needed in PCS \DTRPcs.
    \begin{center}
        \scriptsize
        \begin{tabular}{@{}lcccccc@{}}
            \toprule
            & \multicolumn{3}{c}{Pre-event} & \multicolumn{3}{c}{Event} \\
            \cmidrule(lr){2-4}\cmidrule(lr){5-7}
            & {MXE}      & {ME}       & {MAE}      & {MXE}      & {ME}       & {MAE}      \\
            \midrule
            \BLOCK{for row in rmPCSI16z3IslandingDeltaP10DeltaQ4}
            {{row[0]}} & {{row[1]}} & {{row[2]}} & {{row[3]}} & {{row[4]}} & {{row[5]}} & {{row[6]}} \\
            \BLOCK{endfor}
            \bottomrule
        \end{tabular}
    \end{center}

    \subsubsection{Compliance checks}

    \noindent Compliance thresholds on the curves:
    \begin{center}
        \scriptsize
        \begin{tabular}{@{}lcccccc@{}}
            \toprule
            & \multicolumn{3}{c}{Pre-event} & \multicolumn{3}{c}{Event} \\
            \cmidrule(lr){2-4}\cmidrule(lr){5-7}
            & {MXE}      & {ME}       & {MAE}      & {MXE}      & {ME}       & {MAE}      \\
            \midrule
            \BLOCK{for row in thmPCSI16z3IslandingDeltaP10DeltaQ4}
            {{row[0]}} & {{row[1]}} & {{row[2]}} & {{row[3]}} & {{row[4]}} & {{row[5]}} & {{row[6]}} \\
            \BLOCK{endfor}
            \bottomrule
        \end{tabular}
    \end{center}

    \noindent Compliance checks on the curves:
    \begin{center}
        \scriptsize
        \begin{tabular}{@{}lccccccc@{}}
            \toprule
            & \multicolumn{3}{c}{Pre-event} & \multicolumn{3}{c}{Event} & \\
            \cmidrule(lr){2-4}\cmidrule(lr){5-7}
            & {MXE}      & {ME}       & {MAE}      & {MXE}      & {ME}       & {MAE}      & Compl.     \\
            \midrule
            \BLOCK{for row in emPCSI16z3IslandingDeltaP10DeltaQ4}
            {{row[0]}} & {{row[1]}} & {{row[2]}} & {{row[3]}} & {{row[4]}} & {{row[5]}} & {{row[6]}} & {{row[7]}} \\
            \BLOCK{endfor}
            \bottomrule
        \end{tabular}
    \end{center}

    \noindent In steady state after the event, the absolute average error must not exceed {{steadystatethreshold}}\% (configured value):
    \begin{center}
        \scriptsize
        \begin{tabular}{cllc}
            \toprule
            Variable   & MAE        & Final Error & Compliance \\
            \midrule
            \BLOCK{for row in ssemPCSI16z3IslandingDeltaP10DeltaQ4}
            {{row[0]}} & {{row[1]}} & {{row[2]}}  & {{row[3]}} \\
            \BLOCK{endfor}
            \bottomrule
        \end{tabular}
    \end{center}
