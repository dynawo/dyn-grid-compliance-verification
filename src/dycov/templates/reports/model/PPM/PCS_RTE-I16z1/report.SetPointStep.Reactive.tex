    %%%%%%%%%%%%%%%%%%%%%%%%%%%%%%%%%%%%%%%%%%%%%%%%%%%%%%%%%%%%%%%%%%%%%%%%%%%%%%%%%%%%%%%%%%%%%%%%%%%%%%%%%%%%%
    %%%%%%%%%%%%%%%%%%%%%%%%%%%%%%%%%%%%%%% SetPoint %%%%%%%%%%%%%%%%%%%%%%%%%%%%%%%%%%%%%%%%%%%%%%%%%
    %%%%%%%%%%%%%%%%%%%%%%%%%%%%%%%%%%%%%%%%%%%%%%%%%%%%%%%%%%%%%%%%%%%%%%%%%%%%%%%%%%%%%%%%%%%%%%%%%%%%%%%%%%%%%

    \renewcommand{\DTRPcs}{SetPointStep} % DTR pcs definition
    \renewcommand{\DTRPcsLong}{SetPoint Step}
    \renewcommand{\OCname}{Reactive}


    \subsection{Step response to control changes: Q-setpoint drop}

    Checks for compliant behavior of the generating unit under a scenario where a $5\%$
    step decrease is applied to the reactive power setpoint.

    \GridCircuitZone

    \begin{description}
        \item Initial conditions used at the PDR bus:
        \begin{itemize}
            \item $P = P_\text{max\_unite}$
            \item $Q = 0$
            \item $U = U_\text{dim}$
            \item $SCR = 3$
        \end{itemize}
    \end{description}

    \subsubsection{Simulation parameters}

    Solver and parameters used in the simulation:
    \begin{center}
        \begin{tabular}{lc}
            \toprule
           \textbf{Parameter} & \textbf{Value (default)} \\
            \midrule
            \BLOCK{for row in solverPCSI16z1SetPointStepReactive}
            {{row[0]}}         & {{row[1]}}                         \\
            \BLOCK{endfor}
            \bottomrule
        \end{tabular}
    \end{center}

    \GridCurvesZone
    \\[2\baselineskip]
    Go to  {{ linkPCSI16z1SetPointStepReactive }}

    \subsubsection{Analysis of results}

    \begin{center}
        \scriptsize
        \begin{tabular}{lccccccc}
            \toprule
            & \multicolumn{3}{c}{Pre-event} & \multicolumn{3}{c}{Event} & \\
            \cmidrule(lr){2-4}\cmidrule(lr){5-7}
            & {MXE}      & {ME}       & {MAE}      & {MXE}      & {ME}       & {MAE}      \\
            \midrule
            \BLOCK{for row in rmPCSI16z1SetPointStepReactive}
            {{row[0]}} & {{row[1]}} & {{row[2]}} & {{row[3]}} & {{row[4]}} & {{row[5]}} & {{row[6]}} \\
            \BLOCK{endfor}
            \bottomrule
        \end{tabular}
    \end{center}

    \subsubsection{Compliance checks}

    \noindent Compliance thresholds on the curves:
    \begin{center}
        \scriptsize
        \begin{tabular}{lcccccc}
            \toprule
            & \multicolumn{3}{c}{Pre-event} & \multicolumn{3}{c}{Event} \\
            \cmidrule(lr){2-4}\cmidrule(lr){5-7}
            & {MXE}      & {ME}       & {MAE}      & {MXE}      & {ME}       & {MAE}      \\
            \midrule
            \BLOCK{for row in thmPCSI16z1SetPointStepReactive}
            {{row[0]}} & {{row[1]}} & {{row[2]}} & {{row[3]}} & {{row[4]}} & {{row[5]}} & {{row[6]}} \\
            \BLOCK{endfor}
            \bottomrule
        \end{tabular}
    \end{center}

    \noindent Compliance checks on the curves:
    \begin{center}
        \scriptsize
        \begin{tabular}{lccccccc}
            \toprule
            & \multicolumn{3}{c}{Pre-event} & \multicolumn{3}{c}{Event} & \\
            \cmidrule(lr){2-4}\cmidrule(lr){5-7}
            & {MXE}      & {ME}       & {MAE}      & {MXE}      & {ME}       & {MAE}      & Compl.     \\
            \midrule
            \BLOCK{for row in emPCSI16z1SetPointStepReactive}
            {{row[0]}} & {{row[1]}} & {{row[2]}} & {{row[3]}} & {{row[4]}} & {{row[5]}} & {{row[6]}} & {{row[7]}} \\
            \BLOCK{endfor}
            \bottomrule
        \end{tabular}
    \end{center}

    \noindent Compliance checks on the step-response characteristic: \\
    \begin{minipage}{\linewidth} % because otherwise, the footnote does not show
        \centering
        \scriptsize
        \begin{tabular}{lccccc}
            \toprule
            Step-response indicator & Simulated  & Reference  & Rel. Err. (\%) & Threshold (\%) & Compl.     \\
            \midrule
            \BLOCK{for row in temPCSI16z1SetPointStepReactive}
            {{row[0]}}              & {{row[1]}} & {{row[2]}} & {{row[3]}} & {{row[4]}} & {{row[5]}} \\
            \BLOCK{endfor}
            \bottomrule
        \end{tabular}
    \end{minipage}

    \noindent In steady state after the event, the absolute average error must not exceed {{steadystatethreshold}}\% (configured value):
    \begin{center}
        \scriptsize
        \begin{tabular}{cllc}
            \toprule
            Variable   & MAE        & Final Error & Compliance \\
            \midrule
            \BLOCK{for row in ssemPCSI16z1SetPointStepReactive}
            {{row[0]}} & {{row[1]}} & {{row[2]}}  & {{row[3]}} \\
            \BLOCK{endfor}
            \bottomrule
        \end{tabular}
    \end{center}
