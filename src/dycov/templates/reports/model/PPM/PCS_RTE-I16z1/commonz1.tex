\renewcommand{\DTRPcs}{SetPointStep} % DTR pcs definition
\renewcommand{\OCname}{Active}

%% Our colors (for backgrounds and code listings):
\definecolor{light-gray}{gray}{0.9}
\definecolor{dark-gray}{gray}{0.4}
\definecolor{light-blue}{RGB}{64,64,255}
\definecolor{dark-blue}{RGB}{16,16,64}
\definecolor{dark-green}{RGB}{16,128,16}

%Snippet for the circuits
\newcommand{\GridCircuitZone}{
    The grid model and its operational point is as in the following schematic:
    \begin{center}
        \includestandalone[width=0.8\textwidth]{circuit_z1.tikz}
    \end{center}
    \begin{center}
        \small \textbf{Note: This schematic is only a reminder of the test setup on the TSO's
        side --- the Producer's side may vary, depending on the user-provided model.}
    \end{center}
}
%End snippet for the circuits

%Snippet for the figures
\newcommand{\GridCurvesZone}{

    \subsubsection{Simulation}
    As required by the DTR PCS \DTRPcs, the figures below show the
    following magnitudes:
    \begin{itemize}
        \item Voltage at connection point: modulus of the complex AC voltage at
        the PDR bus.
        \item Active power supplied at the connection point: sum of active power
        over all lines on TSO side.
        \item Reactive power supplied at the connection point: sum of reactive power
        over all lines on TSO side.
        \item Active current supplied at the connection point: sum of active power
        over all lines on TSO side.
        \item Reactive current supplied at the connection point: sum of reactive power
        over all lines on TSO side.
        \item Magnitude controlled by the AVR: modulus of the voltage at the REPC.
        All PPM units are plotted on the same graph.
    \end{itemize}

    \subsubsection{Simulation results}
    The blue line shows the calculated curve and the orange line shows the reference curve.

% For now we won't use floats for figures, to get more precise placement
    \noindent
    \begin{minipage}[t]{0.48\textwidth}
        \centering
        \includegraphics[width=\textwidth]{fig_V_PCS_RTE-I16z1.\DTRPcs.\OCname}
        \begin{minipage}[t]{0.8\textwidth}
            \small Voltage magnitude measured at the PDR bus.
%     The vertical dashed line marks the rise time $T_{85U}$.
        \end{minipage}
    \end{minipage}
    \hfill
    \begin{minipage}[t]{0.48\textwidth}
        \centering
        \includegraphics[width=\textwidth]{fig_Ustator_PCS_RTE-I16z1.\DTRPcs.\OCname}
        \begin{minipage}[t]{0.8\textwidth}
            \small Magnitude controlled by the AVR, in pu. All PPM
            units are plotted on the same graph. The gray dotted
            line shows the AVR setpoint.
        \end{minipage}
    \end{minipage}
%
    \vspace{0.5cm}
    \begin{minipage}[t]{0.48\textwidth}
        \centering
        \includegraphics[width=\textwidth]{fig_P_PCS_RTE-I16z1.\DTRPcs.\OCname}
        \begin{minipage}[t]{0.8\textwidth}
            \small Real power output P, measured at the PDR bus.
        \end{minipage}
    \end{minipage}
    \hfill
    \begin{minipage}[t]{0.48\textwidth}
        \centering
        \includegraphics[width=\textwidth]{fig_Q_PCS_RTE-I16z1.\DTRPcs.\OCname}
        \begin{minipage}[t]{0.8\textwidth}
            \small Reactive power output Q, measured at the PDR bus.
        \end{minipage}
    \end{minipage}
%
    \vspace{0.5cm}
    \begin{minipage}[t]{0.48\textwidth}
        \centering
        \includegraphics[width=\textwidth]{fig_Ire_PCS_RTE-I16z1.\DTRPcs.\OCname}
        \begin{minipage}[t]{0.8\textwidth}
            \small Active current output Ip, measured at the PDR bus.
        \end{minipage}
    \end{minipage}
    \hfill
    \begin{minipage}[t]{0.48\textwidth}
        \centering
        \includegraphics[width=\textwidth]{fig_Iim_PCS_RTE-I16z1.\DTRPcs.\OCname}
        \begin{minipage}[t]{0.8\textwidth}
            \small Reactive current output Iq, measured at the PDR bus.
        \end{minipage}
    \end{minipage}
}
%End of snippet for the figures
