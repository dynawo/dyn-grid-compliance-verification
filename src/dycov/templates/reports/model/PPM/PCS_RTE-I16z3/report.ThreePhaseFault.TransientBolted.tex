    \renewcommand{\DTRPcs}{ThreePhaseFault} % DTR pcs definition
    \renewcommand{\OCname}{TransientBolted}


    \subsection{Test I5 - Transient fault ride-through}

    Checks for compliant behavior of the PPM under a grid scenario where there
    is a symmetric three-phase fault at one of the four transmission lines
    (at a 1\% distance from the PDR). The PPM should remain connected and
    be able to supply the necessary reactive injection.

    \GridCircuitZthree

    \noindent Important reminders:
    \begin{itemize}
        \item The time for fault clearance, $T_{clear}$, is a parameter that
        is specified in the DTR and depends on the voltage level in
        question:
        \begin{itemize}
            \item HTB3 (400 kV): $T_{clear}$ = 85 ms
            \item HTB2 (150 kV and 225 kV): $T_{clear}$ = 85 ms
            \item HTB1 (63 kV and 90 kV): $T_{clear}$ = 150 ms
        \end{itemize}
    \end{itemize}

    In \Dynawo this fault has been modeled by splitting Line4 into Line4a
    and Line4b, and then inserting an intermediate bus between them, to
    which a NodeFault model is attached.

    Note that although the DTR PCS also requires simulating a one-phase
    fault, this tool will not implement it, as it would require the
    simulation of a three-phase model. All simulations are only
    positive-sequence.

    \begin{description}
        \item Initial conditions used at the PDR bus:
        \begin{itemize}
            \item $P = P_{max\_unite}$
            \item $Q = 0$
            \item $U = U_{dim}$
        \end{itemize}
    \end{description}

    \subsubsection{Simulation parameters}

    Solver and parameters used in the simulation:
    \begin{center}
        \begin{tabular}{lc}
            \toprule
           \textbf{Parameter} & \textbf{Value (default)} \\
            \midrule
            \BLOCK{for row in solverPCSI16z3ThreePhaseFaultTransientBolted}
            {{row[0]}}         & {{row[1]}}                         \\
            \BLOCK{endfor}
            \bottomrule
        \end{tabular}
    \end{center}

    \subsubsection{Simulation}
    As required by the DTR PCS \DTRPcs, the figures below show the
    following magnitudes:
    \begin{itemize}
        \item Voltage at connection point: modulus of the complex AC voltage at
        the PDR bus.
        \item Active power supplied at the connection point: sum of active power
        over all lines on TSO side.
        \item Reactive power supplied at the connection point: sum of reactive power
        over all lines on TSO side.
        \item Active current supplied at the connection point: sum of active power
        over all lines on TSO side.
        \item Reactive current supplied at the connection point: sum of reactive power
        over all lines on TSO side.
        \item The injected active and reactive currents. All PPM
        units are plotted on the same graph.
        \item Magnitude controlled by the AVR: modulus of the voltage at the REPC.
        All PPM units are plotted on the same graph.
        \item The PPM's main transformer tap ratio, if it is a transformer with an
        automatic tap changer.
    \end{itemize}

    \subsubsection{Simulation results}
    % For now we won't use floats for figures, to get more precise placement
    \noindent
    \begin{minipage}[t]{0.48\textwidth}
        \centering
        \includegraphics[width=\textwidth]{fig_V_PCS_RTE-I16z3.\DTRPcs.\OCname}
        \begin{minipage}[t]{0.8\textwidth}
            \small Voltage magnitude measured at the PDR bus.
    % The vertical dashed line marks the rise time $T_{85U}$.
        \end{minipage}
    \end{minipage}
    \hfill
    \begin{minipage}[t]{0.48\textwidth}
        \centering
        \includegraphics[width=\textwidth]{fig_Ustator_PCS_RTE-I16z3.\DTRPcs.\OCname}
        \begin{minipage}[t]{0.8\textwidth}
            \small Magnitude controlled by the AVR: modulus of the voltage
            at the REPC, in pu. All PPM units are plotted on the same graph.
            The gray dotted line show the AVR setpoint.
        \end{minipage}
    \end{minipage}

    \vspace{0.5cm}
    \begin{minipage}[t]{0.48\textwidth}
        \centering
        \includegraphics[width=\textwidth]{fig_P_PCS_RTE-I16z3.\DTRPcs.\OCname}
        \begin{minipage}[t]{0.8\textwidth}
            \small Real power output P, measured at the PDR bus.
        \end{minipage}
    \end{minipage}
    \hfill
    \begin{minipage}[t]{0.48\textwidth}
        \centering
        \includegraphics[width=\textwidth]{fig_Q_PCS_RTE-I16z3.\DTRPcs.\OCname}
        \begin{minipage}[t]{0.8\textwidth}
            \small Reactive power output Q, measured at the PDR bus.
        \end{minipage}
    \end{minipage}

    \vspace{0.5cm}
    \begin{minipage}[t]{0.48\textwidth}
        \centering
        \includegraphics[width=\textwidth]{fig_Ire_PCS_RTE-I16z3.\DTRPcs.\OCname}
        \begin{minipage}[t]{0.8\textwidth}
            \small Active current output Ip,measured at the PDR bus.
        \end{minipage}
    \end{minipage}
    \hfill
    \begin{minipage}[t]{0.48\textwidth}
        \centering
        \includegraphics[width=\textwidth]{fig_Iim_PCS_RTE-I16z3.\DTRPcs.\OCname}
        \begin{minipage}[t]{0.8\textwidth}
            \small Reactive current output Iq,measured at the PDR bus.
        \end{minipage}
    \end{minipage}

    \vspace{0.5cm}
    \begin{minipage}[t]{0.48\textwidth}
        \centering
        \includegraphics[width=\textwidth]{fig_Tap_PCS_RTE-I16z3.\DTRPcs.\OCname}
        \begin{minipage}[t]{0.8\textwidth}
            \small The PPM's main transformer tap ratio (if it is a transformer with an
            automatic tap changer).
        \end{minipage}
    \end{minipage}
    \\[2\baselineskip]
    Go to  {{ linkPCSI16z3ThreePhaseFaultTransientBolted }}

    \subsubsection{Analysis of results}

    \noindent Analysis of results for PCS \DTRPcs. Values extracted
    from the simulation:

    \begin{center}
        \scriptsize
        \begin{tabular}{@{}lccccccccc@{}}
            \toprule
            & \multicolumn{3}{c}{Pre-event} & \multicolumn{3}{c}{Event} & \multicolumn{3}{c}{Pos-Event} \\
            \cmidrule(lr){2-4}\cmidrule(lr){5-7}\cmidrule(lr){8-10}
            & {MXE}      & {ME}       & {MAE}      & {MXE}      & {ME}       & {MAE}      & {MXE}      & {ME}       & {MAE}      \\
            \midrule
            \BLOCK{for row in rmPCSI16z3ThreePhaseFaultTransientBolted}
            {{row[0]}} & {{row[1]}} & {{row[2]}} & {{row[3]}} & {{row[4]}} & {{row[5]}} & {{row[6]}} & {{row[7]}} & {{row[8]}} & {{row[9]}} \\
            \BLOCK{endfor}
            \bottomrule
        \end{tabular}
    \end{center}

    \subsubsection{Compliance checks}

    \noindent Compliance thresholds on the curves:
    \begin{center}
        \scriptsize
        \begin{tabular}{@{}lccccccccc@{}}
            \toprule
            & \multicolumn{3}{c}{Pre-event} & \multicolumn{3}{c}{Event} & \multicolumn{3}{c}{Pos-Event} \\
            \cmidrule(lr){2-4}\cmidrule(lr){5-7}\cmidrule(lr){8-10}
            & {MXE}      & {ME}       & {MAE}      & {MXE}      & {ME}       & {MAE}      & {MXE}      & {ME}       & {MAE}      \\
            \midrule
            \BLOCK{for row in thmPCSI16z3ThreePhaseFaultTransientBolted}
            {{row[0]}} & {{row[1]}} & {{row[2]}} & {{row[3]}} & {{row[4]}} & {{row[5]}} & {{row[6]}} & {{row[7]}} & {{row[8]}} & {{row[9]}} \\
            \BLOCK{endfor}
            \bottomrule
        \end{tabular}
    \end{center}

    \noindent Compliance checks on the curves:
    \begin{center}
        \scriptsize
        \begin{tabular}{@{}lcccccccccc@{}}
            \toprule
            & \multicolumn{3}{c}{Pre-event} & \multicolumn{3}{c}{Event} & \multicolumn{3}{c}{Pos-Event} & \\
            \cmidrule(lr){2-4}\cmidrule(lr){5-7}\cmidrule(lr){8-10}
            & {MXE}      & {ME}       & {MAE}      & {MXE}      & {ME}       & {MAE}      & {MXE}      & {ME}       & {MAE}      & Compl.      \\
            \midrule
            \BLOCK{for row in emPCSI16z3ThreePhaseFaultTransientBolted}
            {{row[0]}} & {{row[1]}} & {{row[2]}} & {{row[3]}} & {{row[4]}} & {{row[5]}} & {{row[6]}} & {{row[7]}} & {{row[8]}} & {{row[9]}} & {{row[10]}} \\
            \BLOCK{endfor}
            \bottomrule
        \end{tabular}
    \end{center}

    \noindent Compliance checks on the Active Power Recovery:
    \begin{center}
        \scriptsize
        \begin{tabular}{cllc}
            \toprule
            Variable   & Error      & Threshold   & Check      \\
            \midrule
            \BLOCK{for row in aprPCSI16z3ThreePhaseFaultTransientBolted}
            {{row[0]}} & {{row[1]}} & {{row[2]}}  & {{row[3]}} \\
            \BLOCK{endfor}
            \bottomrule
        \end{tabular}
    \end{center}

    \noindent In steady state after the event, the absolute average error must not exceed {{steadystatethreshold}}\% (configured value):
    \begin{center}
        \scriptsize
        \begin{tabular}{cllc}
            \toprule
            Variable   & MAE        & Final Error & Compliance \\
            \midrule
            \BLOCK{for row in ssemPCSI16z3ThreePhaseFaultTransientBolted}
            {{row[0]}} & {{row[1]}} & {{row[2]}}  & {{row[3]}} \\
            \BLOCK{endfor}
            \bottomrule
        \end{tabular}
    \end{center}
